\documentclass[12pt]{article}
\usepackage{amsmath} 
\usepackage{amsfonts}  
\usepackage{geometry} \geometry{a4paper, margin=1in}
\usepackage{hyperref}  
\usepackage{cite}
\usepackage{titlesec}
\usepackage{graphicx}

\titleformat{\section}
  {\normalfont\Large\bfseries\centering}
  {\thesection}
  {1em}
  {}
\title{Covariance and Fitting of Exponential Curve}
\date{\today}

\begin{document}
\maketitle

\begin{abstract}
Covariance measures how two variables change together. In exponential curve fitting, we estimate parameters (like \( a \) and \( b \)) to best fit the data. The covariance between these parameters shows how changes in one affect the other. 
\end{abstract}

\providecommand{\keywords}[1]{\textbf{\textit{Keywords:}} #1}
\keywords{Correlation, Curve Fitting, Variance} 

\section*{Covariance}
\subsection*{Definition}
Covariance is a statistical measure that indicates the extent to which two random variables change together. If the variables tend to increase or decrease together, the covariance is positive. If one variable increases while the other decreases, the covariance is negative. A covariance of zero indicates no relationship between the variables.
\subsection*{Covariance Theorem and Formula}

The covariance between two random variables \( X \) and \( Y \) is given by:
\[
\text{Cov}(X, Y) = \mathbb{E}[(X] - \mathbb{E}[X])(Y - \mathbb{E}[Y])]
\]
Alternatively, this can be expressed as:
\[
\text{Cov}(X, Y) = \mathbb{E}[XY] - \mathbb{E}[X] \cdot \mathbb{E}[Y]
\]
Where:
\begin{itemize}
    \item \( \mathbb{E}[X] \) is the expected value (mean) of \( X \),
    \item \( \mathbb{E}[Y] \) is the expected value (mean) of \( Y \),
    \item \( \mathbb{E}[XY] \) is the expected value of the product of \( X \) and \( Y \).
\end{itemize}

\subsubsection*{\hspace{0.5cm}Example 01:}
     Find the covariance for the following data: 
\subsubsection*{\hspace{0.5cm}Solution:}
\( x = \{6, 5, 3, 4, 2\} \), \( y = \{12, 10, 8, 6, 4\} \), and \( N = 5 \)

Step 1: Calculate Means
\[
\bar{X} = \frac{6 + 5 + 3 + 4 + 2}{5} = \frac{20}{5} = 4
\]
\[
\bar{Y} = \frac{12 + 10 + 8 + 6 + 4}{5} = \frac{40}{5} = 8
\]
Step 2: Using the Sample Covariance Formula
\[
\text{Cov}(x, y) = \frac{\sum (x_i - \bar{X}) (y_i - \bar{Y})}{N - 1}
\]
\[
= \frac{(6 - 4)(12 - 8) + (5 - 4)(10 - 8) + (3 - 4)(8 - 8) + (4 - 4)(6 - 8) + (2 - 4)(4 - 8)}{5 - 1}
\]
\[
= \frac{8 + 2 + 0 + 0 + 6}{4}
\]
\[
= \frac{16}{4} = 4
\]
Step 3: Using the Population Covariance Formula:
\[
\text{Cov}(x, y) = \frac{\sum (x_i - \bar{X}) (y_i - \bar{Y})}{N}
\]
\[
= \frac{(6 - 4)(12 - 8) + (5 - 4)(10 - 8) + (3 - 4)(8 - 8) + (4 - 4)(6 - 8) + (2 - 4)(4 - 8)}{5}
\]
\[
= \frac{8 + 2 + 0 + 0 + 6}{5}
\]
\[
= \frac{16}{5} = 3.2
\]
\subsubsection*{\hspace{0.5cm}Example 02:}
The standard deviation of \(X\) is 15, and \(Y\) is 20. The correlation between \(X\) and \(Y\) is 5. Find the covariance of \(X\) and \(Y\).
\subsubsection*{\hspace{0.5cm}Solution:}
Given Data
\[
\sigma_X = 15, \quad \sigma_Y = 20, \quad \rho(X, Y) = 5
\]

The formula for the correlation between \(X\) and \(Y\) is:
\[
\rho(X, Y) = \frac{\text{Cov}(X, Y)}{\sigma_X \sigma_Y}
\]
{Substitute the Values}
\[
5 = \frac{\text{Cov}(X, Y)}{15 \times 20}
\]

Solve for Covariance
\[
\text{Cov}(X, Y) = 5 \times 15 \times 20 = 1500
\]

Thus, the covariance of \(X\) and \(Y\) is:
\[
\boxed{1500}
\]

\subsubsection*{\hspace{0.5cm}Example 03:}
Compute the covariance for the following data set:
\subsubsection*{\hspace{0.5cm}Solution:}
\[
X = \{10, 12, 15, 18, 20\}, \quad Y = \{8, 10, 12, 14, 16\}
\]

Step 1: Calculate the mean of \(X\) and \(Y\)

\[
\bar{X} = \frac{10 + 12 + 15 + 18 + 20}{5} = \frac{75}{5} = 15
\]
\[
\bar{Y} = \frac{8 + 10 + 12 + 14 + 16}{5} = \frac{60}{5} = 12
\]

Step 2: Compute the deviations from the mean and their products

\[
\begin{array}{|c|c|c|c|c|}
\hline
X & Y & X - \bar{X} & Y - \bar{Y} & (X - \bar{X})(Y - \bar{Y}) \\
\hline
10 & 8  & -5  & -4  & 20 \\
12 & 10 & -3  & -2  & 6 \\
15 & 12 & 0   & 0   & 0 \\
18 & 14 & 3   & 2   & 6 \\
20 & 16 & 5   & 4   & 20 \\
\hline
\end{array}
\]

Step 3: Sum the products

\[
\sum (X - \bar{X})(Y - \bar{Y}) = 20 + 6 + 0 + 6 + 20 = 52
\]

Step 4: Calculate thesample covariance
For \(n = 5\)
\[
\text{Cov}(X, Y) = \frac{\sum (X - \bar{X})(Y - \bar{Y})}{n - 1} = \frac{52}{5 - 1} = \frac{52}{4} = 13
\]
The covariance of the given data set is:
\[
\boxed{13}
\]
Since the covariance is positive, this indicates that \(X\) and \(Y\) tend to increase together.

\section*{Fitting of Exponential Curve}
\subsection*{Definition}
Exponential curve fitting involves finding an exponential function of the form:
\[
y = a e^{bx}
\]
that best fits a set of data points. It is used to model data that grows or decays at a constant rate, and is applied in fields such as biology, finance, and physics to identify patterns and trends.

\subsection*{Theorem and Formulae}
Exponential curve fitting is a method used to find an exponential function that best approximates the relationship between two variables, typically represented as:
\[
y = a e^{bx}
\]
where \(a\) and \(b\) are constants, and \(e\) is the base of the natural logarithm. This method is commonly applied when data exhibits exponential growth or decay. 
By transforming the exponential function into a linear form using logarithms, we can apply linear regression to estimate the parameters \(a\) and \(b\).
Given a set of \(n\) data points \((x_1, y_1), (x_2, y_2), \ldots, (x_n, y_n)\), we transform the equation \(y = a e^{bx}\) by taking the natural logarithm of both sides:

\[
\ln y = \ln a + bx
\]

This becomes a linear equation:
\[
Y = A + bx
\]
where \(Y = \ln y\) and \(A = \ln a\). The goal is to find the best-fitting values for \(A\) and \(b\) using linear regression. The parameters are determined by:
\[
b = \frac{n\sum x_i \ln y_i - \sum x_i \sum \ln y_i}{n\sum x_i^2 - (\sum x_i)^2}
\]
\[
A = \frac{\sum \ln y_i - b \sum x_i}{n}
\]
Once \(A\) is known, \(a\) can be found as:
\[
a = e^A
\]
Thus, the exponential function that best fits the data is:

\[
y = a e^{bx}
\]
\subsubsection*{\hspace{0.5cm}Example 01:}
Calculate the fitting exponential equation \( y = ax^b \) using the least squares method for the following data:
\subsubsection*{\hspace{0.5cm}Solution:}
\[
\begin{array}{|c|c|}
\hline
x & y \\
\hline
2 & 27.8 \\
3 & 62.1 \\
4 & 110 \\
5 & 161 \\
\hline
\end{array}
\]
The curve to be fitted is \( y = ax^b \).

Taking the logarithm on both sides, we get:

\[
\log_{10} y = \log_{10} a + b \log_{10} x
\]

Let \( Y = \log_{10} y \), \( A = \log_{10} a \), and \( X = \log_{10} x \).  
This is now a linear equation in \( Y \) and \( X \):
\[
Y = A + bX
\]
The corresponding normal equations are:
\[
\sum Y = nA + b \sum X
\]
\[
\sum XY = A \sum X + b \sum X^2
\]

The values are calculated using the following table:

\[
\begin{array}{|c|c|c|c|c|c|}
\hline
x & y & X = \log_{10}(x) & Y = \log_{10}(y) & X^2 & X \cdot Y \\
\hline
2 & 27.8 & 0.301 & 1.444 & 0.0906 & 0.4347 \\
3 & 62.1 & 0.4771 & 1.7931 & 0.2276 & 0.8555 \\
4 & 110 & 0.6021 & 2.0414 & 0.3625 & 1.229 \\
5 & 161 & 0.699 & 2.2068 & 0.4886 & 1.5425 \\
\hline
\sum x = 14 & \sum y = 360.9 & \sum X = 2.0792 & \sum Y = 7.4854 & \sum X^2 = 1.1693 & \sum X \cdot Y = 4.0618 \\
\hline
\end{array}
\]

Substituting these values in the normal equations:

\[
4A + 2.0792b = 7.4854
\]
\[
2.0792A + 1.1693b = 4.0618
\]

Solving these equations, we get:

\[
A = 0.868, \quad b = 1.9303
\]

Thus, \( a = \text{antilog}_{10}(A) = \text{antilog}_{10}(0.868) = 7.3791 \).

Now, substituting these values in the equation \( y = ax^b \), we get:

\[
y = 7.3791 \cdot x^{1.9303}
\]

\subsubsection*{\hspace{0.5cm}Example 02:}
Given Data:
Calculate the fitting exponential equation \(y = ax^b\) for the following data:
\subsubsection*{\hspace{0.5cm}Solution:}
\[
\begin{array}{|c|c|}
\hline
x & y \\
\hline
1 & 2.7 \\
2 & 7.4 \\
3 & 20.1 \\
4 & 54.6 \\
5 & 148.4 \\
\hline
\end{array}
\]

Step 1: Transform the Data
Taking the logarithm on both sides of the equation:

\[
\log_{10} y = \log_{10} a + b \log_{10} x
\]

Let \(Y = \log_{10} y\), \(A = \log_{10} a\), and \(X = \log_{10} x\). The equation can now be expressed as:
\[
Y = A + bX
\]
Step 2: Prepare the Table
We calculate \(X\) and \(Y\) values:
\[
\begin{array}{|c|c|c|c|c|c|}
\hline
x & y & X = \log_{10}(x) & Y = \log_{10}(y) & X^2 & XY \\
\hline
1 & 2.7 & 0 & 0.4314 & 0 & 0 \\
2 & 7.4 & 0.3010 & 0.8686 & 0.0906 & 0.2618 \\
3 & 20.1 & 0.4771 & 1.3032 & 0.2276 & 0.6200 \\
4 & 54.6 & 0.6021 & 1.7370 & 0.3625 & 1.0435 \\
5 & 148.4 & 0.6990 & 2.1716 & 0.4886 & 1.5195 \\
\hline
\sum x = 15 & \sum y = 233.2 & \sum X = 2.0792 & \sum Y = 6.5118 & \sum X^2 = 1.1693 & \sum XY = 3.4448 \\
\hline
\end{array}
\]
Step 3: Calculate Normal Equations


\[
\sum Y = nA + b \sum X
\]
\[
\sum XY = A \sum X + b \sum X^2
\]
Substituting the sums into the equations:
1. For \(n = 5\):
\[
6.5118 = 5A + b(2.0792) \tag{1}
\]
2. For the second equation:
\[
3.4448 = A(2.0792) + b(1.1693) \tag{2}
\]
Step 4: Solve the System of Equations
From Equation (1):
\[
5A + 2.0792b = 6.5118
\]
From Equation (2):
\[
2.0792A + 1.1693b = 3.4448
\]
Using substitution or elimination methods, we  solve these equations to find \(A\) and \(b\).
After solving, we obtain:
\[
A \approx 0.6692, \quad b \approx 1.4160
\]
Step 5: Find \(a\)
Using the relationship \(a = 10^A\):
\[
a \approx 10^{0.6692} \approx 4.688
\]
Step 6: Write the Exponential Model
Now substituting the values of \(a\) and \(b\) back into the exponential equation:
\[
y \approx 4.688 \cdot x^{1.4160}
\]
The fitted exponential curve is given by:
\[
y \approx 4.688 \cdot x^{1.4160}
\]
This model effectively captures the growth behavior of the dataset.
\subsubsection*{\hspace{0.5cm}Example 03:}
Fit an exponential curve of the type \( Y = ae^{bX} \) from the following data:
\[
\begin{array}{|c|c|}
\hline
x & y \\
\hline
1 & 5 \\
2 & 10 \\
4 & 30 \\
\hline
\end{array}
\]
\subsubsection*{\hspace{0.5cm}Solution:}
To fit the exponential curve \( Y = ae^{bX} \), we first take the logarithm of both sides:
\[
\log Y = \log a + bX
\]

Let \( u = \log Y \) and \( A = \log a \). The equation can now be expressed as:
\[
u = A + bX
\]

The normal equations are:

\[
\sum u = nA + B\sum x^2
\]
\[
\sum ux = A\sum x + B\sum x^2
\]
Now, let's prepare the table for the computations:
\[
\begin{array}{|c|c|c|c|c|}
\hline
x & y & u = \log y & x & x^2 \\
\hline
1 & 5 & 0.6990 & 1 & 1 \\
2 & 10 & 1.0000 & 2 & 4 \\
4 & 30 & 1.4771 & 4 & 16 \\
\hline
\sum x = 7 & \sum y = 45 & \sum u = 3.1761 & \sum x = 7 & \sum x^2 = 21 \\
\hline
\end{array}
\]

Now, substituting the values
1. From the first equation:

\[
3A + 7B = 3.1761 \tag{1}
\]

2. From the second equation:

\[
7A + 21B = 8.6075 \tag{2}
\]

By solving we get:

\[
A \approx 0.4604 \quad \text{and} \quad B \approx 0.2564
\]

Next, we calculate \( a \):
\[
a = \text{antilog}(A) = \text{antilog}(0.4604) \approx 2.8867
\]

For \( b \):
\[
b = \frac{0.2564}{\log(2.71828)} \approx 0.5904
\]
Thus, the curve of best fit is:
\[
Y = 2.8867 e^{0.5904X}
\]
\subsection*{Conclusion}
Covariance measures the relationship between two variables, indicating whether they move together or inversely. In curve fitting, covariance helps assess how well one variable predicts another, guiding the fit of mathematical models to data. Together, they are essential for understanding and modeling relationships in data, helping to identify trends and make accurate predictions.

\cite{veerarajan2008probability,grewal2003higher,9647}
\bibliographystyle{unsrt}
\bibliography{references}

\end{document}